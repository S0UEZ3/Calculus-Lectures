\documentclass{article}

\usepackage[utf8x]{inputenc}
\usepackage[english,russian]{babel}

\usepackage{graphicx} % Required for inserting images
\usepackage{amsmath, amssymb, amsthm} 
\usepackage[letterpaper, top=1in, bottom=1.0in, left=1.20in, right=1.20in,heightrounded]{geometry}

\title{Mathematical Analysis Vol.1}
\author{@Souez3}
\date{22.11.2024}

\begin{document}

\maketitle

\section{Билет 1}
\subsection{Последовательность}

${f(n)}$ - последовательность задана на множестве N
Когда каждому $n\in \mathds{N}$ поставлено в соответствие некоторого закона
$a(n)\in \mathds{R}$, тогда говорят, что задана числовая последовательность ${a_n}^\inf$

Примеры: 
n-ный член арифметической прогрессии: $a_n = a_1 + \alpha(n-1)$
геометрическая прогрессия: $b_n = b_1 * q^(n-1)$

\subsection{Предел числовой последовательности}
\textbf{Определение:} Число А называют пределом числовой
последовательности ${X_n}$, если $\forall \epsilon > 0 \exists N(\epsilon) : \forall n > N(\epsilon)$ выполняется $|X_n-A| < \epsilon$

\textbf{Определение:} Сходящаяся последовательность - последовательность, которая имеет конечный предел

\textbf{Определение:} Расходящаяся последовательность - последовательность, которая имеет бесконечный предел либо предела не существует.

Последовательноть ограничена, если $\exists M > 0 : \forall n \in \mathds{N}$ выполняется ${a_n} <= M$
(существует такое число М, что для любого номера последовательности все члены последовательности не превосходят это число по модулю.

\section{Билет 2}
\subsection{Теорема о единственности предела последовательности}

\textbf{Теорема:} Если у последовательности есть предел, то он единственный

\textbf{Доказательство:} 
Докажем от противного. Допустим существует 2 предела.
\begin{equation}
    $\sqsupset \lim_{x \to \infty}{X_n} = A$
    $\sqsupset \lim_{x \to \infty}{X_n} = B$, при этом $B!=A$
\end{equation}

Тогда возьмем $\epsilon = (B-A)/3 > 0$, $(\epsilon_A \bigcap \epsilon_B != 0)$

Следовательно
\begin{equation}
    Для $n >= N \exists номер N_1 : \forall n > N$ выполняется $|{X_n} - A| < \epsilon$
\end{equation}
 
\begin{equation}
    Также $\exists N_2 для \forall_n >= N_2$ и тоже выполняется, что $|{X_n} - B| < \epsilon$
\end{equation}
Тогда $|a - b| = |a - {X_n} + {X_n} - b| <= |{X_n} - A| + {X_n} - B| < \epsilon + \epsilon = 2\epsilon = \frac{2*|A-B|}{3}$, тогда получим $|A-B| <= \frac{2}{3}*|B-A|$ 
Получим противорчие

\section{Билет 3}

\textbf{Определение:} Последовательность ограничена, если $\exists M > 0 : \forall b \in \mathds{N}$ выполняется $|a_n| <= M$

\textbf{Теорема об ограниченности сходящейся последовательности:} Всякая сходящаяся последовательность ограничена!

\textbf{Доказательство:} $\sqsupset A = \lim_{n \to \infty}{X_n} \in \mathds{R}$, тогда и только тогда, когда $\forall \epsilon > 0 \exists N(\epsilon) \in mathds{N}$ такое что $\forall n \in mathds{N} : n > N(\epsilon)$ выполняется $|{X_n} - A| < \epsilon \forall n > N(\epsilon) {X_n} \in (A-\epsilon; A+\epsilon)$ содержит конечное число x_1,x_2,...x_k
$\sqsupset m = min{X^-; A-\epsilon} и M = max{A-\epsilon; x^+}$ 
Тогда на отрезке [m;M] находятся x_1,x_2,...x_k и (A-\epsilon;A+\epsilon) следовательно [m;M] содержат все точки {x_n}, то $\forall n \in mathds{N} x_n<=m x_n>=M$

Примеры:
\begin{equation}
    1) ${\frac{1}{n^2}} = {1; \frac{1}{4}; \frac{1}{4}; \frac{1}{9}; \frac{1}{16}...}$ $\lim{\frac{1}{n^2}} = 0$ - ограничена сверху
    2) ${\frac{n^2}{n+1}} = {\frac{1}{2}; \frac{4}{3}; \frac{9}{4}; \frac{16}{5};...}$ $\lim{\frac{n^2}{n+1}} >= \frac{1}{2}$ - ограничена снизу
\end{equation}

\section{Билет 4}
Арифметические операции над сходящимися последовательностями

$\sqsupset{X_n}; {Y_n}$ - две сходящиеся последовательности. Тогда $\exists \lim_{n \to \infty}{X_n} = A; \lim_{n \to \infty}{Y_n} = B$

Cвойства
1) ${X_n += Y_n}; {{X_n} * {Y_n}}; {\frac{X_n}{Y_n}}$ - тоже сходящиеся последовательности.
2) $\lim_{n \to \infty}({X_n} + {Y_n}) = A+B$
3) $\lim_{n \to \infty}({X_n} - {Y_n}) = A-B$
4) $\lim_{n \to \infty}({X_n} * {Y_n}) = A*B$
5) $\lim_{n \to \infty}\frac{X_n}{Y_n} = \frac{A}{B}$

\textbf{Доказательство:} 
1) $\forall N > 0 \existsN_0 : \forall n > N_0$ выполняется $|X_n - A| < \frac{\epsilon}{2}(любая окрестность)$
$\exists N_1 : \forall n > N_1$ выполняется $|{Y_n}-B| < \frac{\epsilon}{2}$
Пусть N = max(N_2; N_1), n > N
Тогда $\forall n > N |({X_n}+{Y_n}) - (A+B)| = |{X_n}- A + {Y_n} - B| <= |{X_n}- A| + |{Y_n} - B| < \frac{\epsilon}{2} + \frac{\epsilon}{2} = \epsilon$

\section{Билет 5}
\subsection{Понятие функции через последовательность}

Если каждому $x \in X$ по некоторому закону поставлен в соответствии единственный y, то говорят что на множестве X задана функция f
\begin{equation}
    $\forall x \in X \exists! y \in \mathds{R} : f(x) = y$
\end{equation}

\subsection{Предел функции в точке}
\textbf{Определение по Гейне:} $\sqsupset f(x)$ - определена в некоторой проколотой окрестности точки x
\begin{equation}
    $\lim _{x \to x_0}{f(x)} = A$ если $\forall {x_n} \exists \mathring{U}_x_0 > 0$

    $\lim _{x \to x_0}{f(x) - g(x)} > 0$ => $f(x) - g(x) > 0$ по теореме если f(x) имеет предел A и в окрестности (а) принимает значения больше нуля, то A >= 0
\end{equation}

\subsection{Теорема о единственности предела} 
Если функция имеет предел в точке, то он единственнй.

Доказательство от противного:
$\sqsupset \exists {X_n} = \lim_{n \to \infty}{X_n} = A$ и $\lim_{n \to \infty}{X_n} = B$, $A != B; A,B \in \mathds{R}$
Возьмем $\epsilon_n \bigcap \epsilon_b != \empty$, тогда $|f(x) - A| < \frac{\epsilon}{2}; |f(x) - B| < \frac{\epsilon}{2}$
$|A-B| = |A-B+f(x)-f(x)| = |A-f(x)+f(x)-B| <= |A-f(x)| + |B-f(x)| < \frac{\epsilon}{2} + \frac{\epsilon}{2} = \epsilon$
То есть получили $\forall \epsilon > 0 -> |A-B| < \epsilon$

\section{Билет 6}

\subsection{Ограниченная функция}

\textbf{Определение:} Функция ограничена, если $\exists M > 0 : \forall x \in \mathds{X}$ выполняется $|f(x)| <= M$

\textbf{Определение:} Функция называется ограниченной сверху на x если $\exists M : \forall x \in X$ выполняется $F(x)<M$

\textbf{Определение:} Функция называется ограниченной снизу на x если $\exists M : \forall x \in X$ выполняется $F(x)>M$

\subsubsection{Теорема об ограниченности функции, имеющей предел (конечный)}
Если функция $f(x)$ определена в точке $x_0$ и имеет в точке конечный предел, то она ограничена в некоторой окрестности этой точки.

\[
\exists \lim_{x \to x_0} f(x) = A \iff \forall \varepsilon > 0 \, \exists \delta > 0 : \forall x \in \dot{U}(\delta),
\]
\[ 
|x - x_0| < \delta \implies |f(x) - A| < \varepsilon.
\]

Пусть $\varepsilon = 1$, тогда $\forall x \in \dot{U}(\delta)$:

\[
|f(x) - A| < 1,
\]

раскрыв модуль:

\[
-1 < f(x) - A < 1.
\]

Отсюда:

\[
A - 1 < f(x) < A + 1 \implies f(x) \text{ ограничена}.
\]

\section{Билет 7}

\subsection{Арифметические действия с пределами функции} 

\[
\lim_{x \to x_0} f(x) = A \quad \text{и} \quad \lim_{x \to x_0} \varphi(x) = B.
\]

Тогда:

1. \[
\lim_{x \to x_0} (f(x) + \varphi(x)) = A + B.
\]

2. \[
\lim_{x \to x_0} C \cdot f(x) = C \cdot A, \quad \text{где } C = \text{const}.
\]

3. \[
\lim_{x \to x_0} (f(x) \cdot \varphi(x)) = A \cdot B.
\]

4. Если $B \neq 0$, то:

\[
\lim_{x \to x_0} \frac{f(x)}{\varphi(x)} = \frac{A}{B}.
\]

Условие: $\forall x \in \operatorname{Dom}(\varphi) \quad \varphi(x) \neq 0$.

\subsection*{Доказательство: Арифметическое свойство предела (Сумма)}

\subsection*{Условие}
\[
\lim_{x \to x_0} f(x) = A \quad \text{и} \quad \lim_{x \to x_0} \varphi(x) = B.
\]

\subsection*{Доказательство}
По определению предела:

\[
\lim_{x \to x_0} f(x) = A \iff \forall \varepsilon_1 > 0 \, \exists \delta_1 > 0 : \forall x \in \dot{U}(\delta_1),
\]
\[
|x - x_0| < \delta_1 \implies |f(x) - A| < \varepsilon_1.
\]

\[
\lim_{x \to x_0} \varphi(x) = B \iff \forall \varepsilon_2 > 0 \, \exists \delta_2 > 0 : \forall x \in \dot{U}(\delta_2),
\]
\[
|x - x_0| < \delta_2 \implies |\varphi(x) - B| < \varepsilon_2.
\]

Пусть $\varepsilon = \varepsilon_1 + \varepsilon_2$, и $\delta = \min(\delta_1, \delta_2)$. Тогда:

\[
|f(x) + \varphi(x) - (A + B)| = |f(x) - A + \varphi(x) - B| \leq |f(x) - A| + |\varphi(x) - B|.
\]

Из условий следует:

\[
|f(x) - A| < \varepsilon_1 \quad \text{и} \quad |\varphi(x) - B| < \varepsilon_2.
\]

Таким образом:

\[
|f(x) + \varphi(x) - (A + B)| < \varepsilon_1 + \varepsilon_2 = \varepsilon.
\]

\subsection*{Вывод}
\[
\lim_{x \to x_0} (f(x) + \varphi(x)) = A + B.
\]

\subsection*{Теорема о суперпозиции}

1) \( f(x)\) и \(g(x)\) : \( F(x) = F(f(g(x))) \)

2) \( \lim_{x \to x_0} g(x) = A \)

3) \( \lim_{x \to x_0} f(x) = B \)

Следовательно:
\[
\lim_{x \to x_0} F(f(g(x))) = B
\]

\textbf{Доказательство:}
\(
    \sqsupset x = Dom(g); y = Dom(f)
\)
Тогда по определению предела
\(
    \lim_{x \to x_0} g(x) = A \iff \forall \varepsilon_1 > 0 \exists \delta_1 > 0 : \forall x \in \dot{U}(\delta_1)
\)
\(
    |g(x) - A| < \varepsilon
\)
\(
    \lim_{y \to A} f(y) = B \iff \forall \varepsilon_2 > 0 \exists \varepsilon_1 > 0 : \forall y \in \dot{U}(A)
\)
\(
    |f(y) - B| < \varepsilon_2
\)
Следовательно:
\(
    \forall \varepsilon_2 > 0 \exists \dot{U}_\delta (x_0) > 0 : \forall x \in \dot{U}_\delta (x_0) \implies f(g(x)) \in \dot{U}_\varepsilon_2 (B) \implies B = \lim_{x \to \x_0} f(g(x))
\)
\(
    |f(g(x)) - B| < \varepsilon_2 \implies B = \lim_{x \to x_0} f(g(x))
\)

\section{Билет 8}

\subsection*{Теоремы о пределах функции: о предельном переходе в неравенство}

Рассмотрим неравенство:
\[
a_n \leq b_n
\]
Пусть \( \lim_{n \to \infty} a_n = A \) и \( \lim_{n \to \infty} b_n = B \). Тогда, если \( a_n \leq b_n \) для всех \( n \), то по свойству пределов:
\[
\lim_{n \to \infty} a_n \leq \lim_{n \to \infty} b_n
\]

Следовательно:
\[
A \leq B
\]

\textbf{Доказательство от противного:}
\(
    \sqsupset A > B
\)
Тогда 
\(
    \lim_{x \to x_0} (f(x) - g(x)) = A - B > 0
\)
Из арифметических свойств пределов следует:
\(
    f(x) - g(x) > 0 \implies f(x) > g(x)
\)
Это противоречит условию $f(x) <= g(x)$

\subsection*{Теорема о сжатой функции}


\subsection*{Теорема о сжатой функции}
Пусть \( f(x) \), \( g(x) \) и \( h(x) \) — функции, определенные на множестве \( E \subset \mathbb{R} \) и выполняется неравенство
\[
f(x) \leq h(x) \leq g(x),
\]
и при этом
\[
\lim_{x \to a} f(x) = \lim_{x \to a} g(x) = b,
\]
то
\[
\lim_{x \to a} h(x) = b.
\]

\textbf{Доказательство:}
\begin{equation}
    \lim_{x \to x_0} f(x) = \lim_{x \to x_0} h(x) = C
\end{equation}
Тогда:
\begin{equation}
    \forall \varepsilon_1 > 0 \exists \dot{U}_f (x_0) : \forall x \in \dot{U}_f (x_0)
\end{equation}
\begin{equation}
    |f(x) - C| < \varepsilon_1 \implies -\varepsilon_1 < f(x) - C < \varepsilon_1 \implies C - \varepsilon_1 < f(x) < \varepsilon_1 + C
\end{equation}

\begin{equation}
    \lim_{x \to x_0} h(x) = C
\end{equation}
\begin{equation}
    \forall \varepsilon_2 > 0 \exists \dot{U}_h (x_0) : \forall x \in \dot{U}_h (x_0)
\end{equation}
\begin{equation}
    |h(x) - C| < \varepsilon_2 \implies -\varepsilon_2 < h(x) - C < \varepsilon_2 \implies C - \varepsilon_2 < h(x) < \varepsilon_2 + C
\end{equation}

\begin{equation}
    f(x) <= g(x) <= h(x) \implies C-\varepsilon_1 < f(x) <= g(x) <= h(x) < \varepsilon_2 + C
\end{equation}
Отсюда:
\begin{equation}
    С - \varepsilon_1 < g(x) < \varepsilon_2 + C
    -\varepsilon_1 < g(x) - C < \varepsilon_2
\end{equation}

Пересечём окрестности $\varepsilon_1$ и $\varepsilon_2$ и возьмем $min(-\varepsilon_2; \varepsilon_2)$
Тогда
\begin{equation}
    -\varepsilon_2 < g(x) - С < \varepsilon_2 \implies |g(x) - C| < \varepsilon_2 \implies \lim_{x \to x_0} g(x) = C
\end{equation}

\subsection*{1 замечательный предел}
Рассмотрим предел:
\[
\lim_{x \to 0} \frac{\sin x}{x} = 1
\]

\textbf{Доказательство:}

Рассмотрим односторонние пределы и докажем, что они равны 1. Рассмотрим случай \( x \to +0 \). Отложим этот угол на единичной окружности так, чтобы его вершина совпадала с началом координат, а одна сторона совпадала с осью \( OX \). Пусть \( A \) — точка пересечения второй стороны угла с единичной окружностью, а точка \( B \) — с касательной к этой окружности в точке \( A \). Точка \( C \) — проекция точки \( A \) на ось \( OX \). Очевидно, что:
\[
S_{\triangle OAC} < S_{\text{сектора } OAC} < S_{\triangle OAB}
\]
где \( S \) — площадь. Поскольку \( |OC| = \cos x \), \( |AC| = \sin x \), \( |AB| = \tan x \), то:
\[
\frac{\sin x}{2} < \frac{x}{2} < \frac{\tan x}{2}
\]
Так как при \( x \to +0 \): \( \sin x > 0 \), \( x > 0 \), \( \tan x > 0 \):
\[
\frac{1}{\tan x} < \frac{1}{x} < \frac{1}{\sin x}
\]
Умножаем на \( \sin x \):
\[
\cos x \leq \frac{\sin x}{x} \leq 1
\]
Переходя к пределу:
\[
\lim_{x \to +0} \cos x \leq \lim_{x \to +0} \frac{\sin x}{x} \leq 1
\]
Так как \( \lim_{x \to +0} \cos x = 1 \), то:
\[
\lim_{x \to +0} \frac{\sin x}{x} = 1
\]

Аналогично доказывается для \( x \to -0 \). Следовательно:
\[
\lim_{x \to 0} \frac{\sin x}{x} = 1
\]

\section{Билет 9}

\subsection*{Предел функции на бесконечности}
\textbf{Определение:} Число A называется пределом функции f(x) при $x \to \infty$ если
\(
\forall \varepsilon > 0 \exists \delta = \delta(\varepsilon) > 0 : \forall x \in Dom(f)
\) из \( |x| > М \Rightarrow |f(x) - A| < \varepsilon \).

\section{Билет 10}

\subsection*{Бесконечно большие функции.}

Функция \( f(x) \) называется бесконечно большой при \( x \to x_0 \), если
\[
\forall M > 0 \, \exists \delta > 0 \, \text{такое, что} \, 0 < |x - x_0| < \delta \Rightarrow |f(x)| > M
\]

Пример:
Функция \( f(x) = \frac{1}{x} \) является бесконечно большой при \( x \to 0 \).

\section{Билет 20}

\subsection*{Устойчивость знака непрерывной функции}

\textbf{Теорема:} Пусть $f$ — непрерывная функция на множестве $D \subset \mathbb{R}$, и пусть $c \in D$ такая точка, что $f(c) \neq 0$. Тогда существует окрестность $U(c)$ точки $c$, такая что для всех $x \in U(c) \cap D$ выполняется $f(x) \neq 0$ и знак функции $f$ на $U(c) \cap D$ совпадает со знаком $f(c)$.


\section{Билет 21}

\subsection*{1. Алгебраические функции}
Алгебраические функции — это функции, которые могут быть выражены с использованием конечного числа операций сложения, вычитания, умножения, деления и извлечения корней. Примеры:
\begin{itemize}
    \item линейная функция: $f(x) = ax + b$, где $a, b \in \mathbb{R}$;
    \item квадратичная функция: $f(x) = ax^2 + bx + c$, где $a, b, c \in \mathbb{R}$;
    \item корневая функция: $f(x) = \sqrt[n]{x}$, где $n \in \mathbb{N}, n \geq 2$.
\end{itemize}

\subsection*{2. Трансцендентные функции}
Трансцендентные функции не могут быть выражены в виде конечных комбинаций алгебраических операций. Они включают:
\begin{itemize}
    \item экспоненциальные функции, например, $f(x) = a^x$, где $a > 0, a \neq 1$;
    \item логарифмические функции, например, $f(x) = \ln(x)$ или $f(x) = \log_a(x)$;
    \item тригонометрические функции: $\sin(x)$, $\cos(x)$, $\tan(x)$ и т.д.;
    \item обратные тригонометрические функции: $\arcsin(x)$, $\arccos(x)$ и т.д.;
    \item гиперболические функции: $\sinh(x)$, $\cosh(x)$ и т.д.
\end{itemize}

\subsection*{3. Непрерывность элементарных функций}
Элементарные функции являются непрерывными на своих областях определения. Это означает, что если функция определена в некоторой точке $x_0$ и в её окрестности, то:
\[
\lim_{x \to x_0} f(x) = f(x_0).
\]
Примеры:
\begin{itemize}
    \item Линейные и квадратичные функции непрерывны на всей числовой прямой $\mathbb{R}$.
    \item Тригонометрические функции $\sin(x)$ и $\cos(x)$ непрерывны на $\mathbb{R}$, а $\tan(x)$ — на множестве $\mathbb{R} \setminus \left\{x = \frac{\pi}{2} + k\pi, \, k \in \mathbb{Z}\right\}$.
\end{itemize}



\section{Билет 22}


\section*{Операции над непрерывными функциями и переход к пределу под знаком непрерывной функции}

\subsection*{Операции над непрерывными функциями}
Пусть $f$ и $g$ — функции, непрерывные в точке $x = a$. Тогда следующие функции также непрерывны в точке $a$:
\begin{itemize}
    \item Сумма: $f(x) + g(x)$
    \item Разность: $f(x) - g(x)$
    \item Произведение: $f(x) \cdot g(x)$
    \item Частное: $\frac{f(x)}{g(x)}$, если $g(a) \neq 0$
\end{itemize}

\subsection*{Переход к пределу под знаком непрерывной функции}
Пусть $f$ — непрерывная функция в точке $a$, и пусть $\lim_{x \to a} g(x) = L$. Тогда:
\[
\lim_{x \to a} f(g(x)) = f\left(\lim_{x \to a} g(x)\right) = f(L)
\]

\textbf{Доказательство:} Так как $f$ непрерывна в точке $L$, то по определению непрерывности для любого $\epsilon > 0$ существует $\delta > 0$ такое, что для всех $y$, удовлетворяющих условию $|y - L| < \delta$, выполняется $|f(y) - f(L)| < \epsilon$. Поскольку $\lim_{x \to a} g(x) = L$, существует такое $\delta' > 0$, что для всех $x$, удовлетворяющих условию $|x - a| < \delta'$, выполняется $|g(x) - L| < \delta$. Следовательно, для таких $x$ имеем:
\[
|f(g(x)) - f(L)| < \epsilon
\]
Таким образом, $\lim_{x \to a} f(g(x)) = f(L)$.

\section{Билет 23}

\section*{Теорема о непрерывности сложной функции}

\textbf{Теорема:} Пусть $f$ непрерывна в точке $a$, и $g$ непрерывна в точке $b = f(a)$. Тогда сложная функция $h(x) = g(f(x))$ непрерывна в точке $a$.

\textbf{Доказательство:} Так как $f$ непрерывна в точке $a$, то для любого $\epsilon > 0$ существует $\delta_1 > 0$ такое, что если $|x - a| < \delta_1$, то $|f(x) - f(a)| < \delta_2$, где $\delta_2$ будет определено далее.

Поскольку $g$ непрерывна в точке $b = f(a)$, то для любого $\epsilon > 0$ существует $\delta_2 > 0$ такое, что если $|y - f(a)| < \delta_2$, то $|g(y) - g(f(a))| < \epsilon$.

Теперь, выберем $\delta = \delta_1$. Тогда, если $|x - a| < \delta$, то $|f(x) - f(a)| < \delta_2$, и, следовательно, $|g(f(x)) - g(f(a))| < \epsilon$.

Таким образом, $|h(x) - h(a)| = |g(f(x)) - g(f(a))| < \epsilon$, что доказывает непрерывность $h(x)$ в точке $a$.


\section{Билет 24}
\subsection*{Первый замечательный предел}
\[
\lim_{x \to 0} \frac{\sin x}{x} = 1
\]

\textbf{Следствия:}
\[
\lim_{x \to 0} \frac{\tan x}{x} = 1, \quad \lim_{x \to 0} \frac{\arcsin x}{x} = 1, \quad \lim_{x \to 0} \frac{\arctan x}{x} = 1
\]
\[
\lim_{x \to 0} \frac{1 - \cos x}{x^2/2} = 1
\]

\subsection*{Второй замечательный предел}
\[
\lim_{x \to \infty} \left(1 + \frac{1}{x}\right)^x = e
\]

\textbf{Следствия:}
\[
\lim_{x \to 0} (1 + x)^{1/x} = e
\]
\[
\lim_{k \to +\infty} \left(1 + \frac{1}{k}\right)^k = e
\]
\[
\lim_{x \to 0} \ln(1 + x) = 1
\]
\[
\lim_{x \to 0} \frac{e^x - 1}{x} = 1
\]
\[
\lim_{x \to 0} \frac{a^x - 1}{x} = \ln(a) \quad \text{для} \quad a > 0, \, a \neq 1
\]
\[
\lim_{x \to 0} \frac{\ln(1 + ax)}{ax} = 1
\]

\section{Билет 25}

\section*{Точки разрыва функции и их классификация}

\subsection*{Точки разрыва первого и второго рода}

\subsubsection*{Точка разрыва первого рода}
Точка $x = a$ называется точкой разрыва первого рода, если существуют конечные односторонние пределы, но они не равны:
\[
\lim_{x \to a^-} f(x) \neq \lim_{x \to a^+} f(x)
\]
\textbf{Пример:} Функция $f(x) = \begin{cases} 
1, & x < 0 \\
2, & x \geq 0 
\end{cases}$ имеет точку разрыва первого рода в $x = 0$.

\subsubsection*{Точка разрыва второго рода}
Точка $x = a$ называется точкой разрыва второго рода, если хотя бы один из односторонних пределов равен $\pm \infty$ или не существует:
\[
\lim_{x \to a^-} f(x) = \pm \infty \text{ или } \lim_{x \to a^+} f(x) = \pm \infty
\]
\textbf{Пример:} Функция $f(x) = \frac{1}{x}$ имеет точку разрыва второго рода в $x = 0$.

\subsection*{Точки устранимого и неустранимого разрыва}

\subsubsection*{Точка устранимого разрыва}
Точка $x = a$ называется точкой устранимого разрыва, если существует конечный предел, но функция не определена в этой точке или её значение не равно пределу:
\[
\lim_{x \to a} f(x) \text{ существует, но } f(a) \neq \lim_{x \to a} f(x) \text{ или } f(a) \text{ не определена}
\]
\textbf{Пример:} Функция $f(x) = \begin{cases} 
\frac{\sin x}{x}, & x \neq 0 \\
0, & x = 0 
\end{cases}$ имеет устранимый разрыв в $x = 0$, так как $\lim_{x \to 0} \frac{\sin x}{x} = 1$.

\subsubsection*{Точка неустранимого разрыва}
Точка $x = a$ называется точкой неустранимого разрыва, если невозможно сделать функцию непрерывной в этой точке ни одним способом:
\[
\lim_{x \to a^-} f(x) \neq \lim_{x \to a^+} f(x) \text{ или хотя бы один из этих пределов не существует}
\]
\textbf{Пример:} Функция $f(x) = \begin{cases} 
1, & x < 0 \\
2, & x \geq 0 
\end{cases}$ имеет неустранимый разрыв в $x = 0$.

\section{Билет 26}

\section*{Непрерывность функции на интервале и на отрезке}

\subsection*{Непрерывность на интервале}
Функция \( f(x) \) непрерывна на интервале \( (a, b) \), если она непрерывна в каждой точке этого интервала.

\subsection*{Непрерывность на отрезке}
Функция \( f(x) \) непрерывна на отрезке \( [a, b] \), если она непрерывна на интервале \( (a, b) \) и в точках \( a \) и \( b \) с учетом односторонних пределов:
\[
\lim_{x \to a^+} f(x) = f(a)
\]
\[
\lim_{x \to b^-} f(x) = f(b)
\]

\subsection*{Кусочно-непрерывные функции на отрезке}
Функция называется кусочно-непрерывной на отрезке, если она непрерывна на каждом подотрезке, на который можно разбить исходный отрезок, за исключением, возможно, конечного числа точек разрыва первого рода.

\section{Билет 27}

\section*{Теоремы Больцано — Коши}

\subsection*{Первая теорема Больцано — Коши (о существовании корня)}

\textbf{Теорема:} Если функция \( f \) непрерывна на отрезке \([a, b]\) и \( f(a) \cdot f(b) < 0 \), то существует точка \( c \in (a, b) \), такая что \( f(c) = 0 \).

\textbf{Доказательство:} Поскольку \( f \) непрерывна на \([a, b]\), то по теореме Вейерштрасса она достигает на этом отрезке своих максимума и минимума. Пусть \( f(a) < 0 \) и \( f(b) > 0 \) (случай \( f(a) > 0 \) и \( f(b) < 0 \) рассматривается аналогично).

Рассмотрим множество \( A = \{ x \in [a, b] \mid f(x) \leq 0 \} \). Множество \( A \) непусто, так как \( a \in A \), и ограничено сверху, так как \( b \notin A \). Пусть \( c = \sup A \). Тогда \( a \leq c \leq b \).

Поскольку \( f \) непрерывна, то:
\[
\lim_{x \to c^-} f(x) = f(c) = \lim_{x \to c^+} f(x)
\]

Если \( f(c) = 0 \), то теорема доказана. Если \( f(c) \neq 0 \), то возможны два случая:
1. \( f(c) > 0 \). Тогда для достаточно малых \( \epsilon > 0 \) имеем \( f(c - \epsilon) < 0 \), что противоречит определению \( c \) как точной верхней грани множества \( A \).
2. \( f(c) < 0 \). Тогда для достаточно малых \( \epsilon > 0 \) имеем \( f(c + \epsilon) > 0 \), что также противоречит определению \( c \).

Следовательно, \( f(c) = 0 \).

\section{Билет 28}

\subsection*{Вторая теорема Больцано — Коши (о промежуточном значении непрерывной функции)}

\textbf{Теорема:} Если функция \( f \) непрерывна на отрезке \([a, b]\) и \( f(a) \neq f(b) \), то для любого числа \( y \) между \( f(a) \) и \( f(b) \) существует точка \( c \in (a, b) \), такая что \( f(c) = y \).

\textbf{Доказательство:} Без ограничения общности предположим, что \( f(a) < y < f(b) \) (случай \( f(a) > y > f(b) \) рассматривается аналогично).

Рассмотрим функцию \( g(x) = f(x) - y \). Функция \( g \) непрерывна на \([a, b]\), и \( g(a) = f(a) - y < 0 \) и \( g(b) = f(b) - y > 0 \).

По первой теореме Больцано — Коши существует точка \( c \in (a, b) \), такая что \( g(c) = 0 \), то есть \( f(c) = y \).

\section*{Теоремы Вейерштрасса}

\subsection*{Первая теорема Вейерштрасса (об ограниченности непрерывной функции)}

\textbf{Теорема:} Всякая функция, непрерывная на отрезке \([a, b]\), ограничена на этом отрезке.

\textbf{Доказательство:} Пусть \( f \) непрерывна на \([a, b]\). Предположим противное, что \( f \) не ограничена на \([a, b]\). Тогда для любого \( n \in \mathbb{N} \) существует \( x_n \in [a, b] \), такое что \( |f(x_n)| > n \). Последовательность \( \{x_n\} \) ограничена, поэтому по теореме Больцано-Вейерштрасса из неё можно выделить сходящуюся подпоследовательность \( \{x_{n_k}\} \), сходящуюся к некоторому \( c \in [a, b] \).

Так как \( f \) непрерывна в точке \( c \), то \( f(x_{n_k}) \to f(c) \) при \( k \to \infty \). Но \( |f(x_{n_k})| > n_k \), что стремится к бесконечности при \( k \to \infty \). Это противоречие доказывает, что \( f \) ограничена на \([a, b]\).

\subsection*{Вторая теорема Вейерштрасса (о наибольшем и наименьшем значении функции на отрезке)}

\textbf{Теорема:} Всякая функция, непрерывная на отрезке \([a, b]\), достигает на этом отрезке своих наибольшего и наименьшего значений.

\textbf{Доказательство:} Пусть \( f \) непрерывна на \([a, b]\). По первой теореме Вейерштрасса \( f \) ограничена на \([a, b]\), то есть существует \( M > 0 \) такое, что \( |f(x)| \leq M \) для всех \( x \in [a, b] \).

Рассмотрим множество \( A = \{ f(x) \mid x \in [a, b] \} \). Множество \( A \) ограничено и по теореме Вейерштрасса о супремуме и инфимуме существует \( \sup A \) и \( \inf A \). Пусть \( \sup A = M \) и \( \inf A = m \).

Так как \( M \) — точная верхняя грань множества \( A \), то существует последовательность \( \{x_n\} \subset [a, b] \), такая что \( f(x_n) \to M \) при \( n \to \infty \). Последовательность \( \{x_n\} \) ограничена, поэтому по теореме Больцано-Вейерштрасса из неё можно выделить сходящуюся подпоследовательность \( \{x_{n_k}\} \), сходящуюся к некоторому \( c \in [a, b] \).

Так как \( f \) непрерывна в точке \( c \), то \( f(x_{n_k}) \to f(c) \) при \( k \to \infty \). Следовательно, \( f(c) = M \), то есть \( f \) достигает своего наибольшего значения \( M \) в точке \( c \).

Аналогично доказывается, что \( f \) достигает своего наименьшего значения \( m \) в некоторой точке \( d \in [a, b] \).

\section{Билет 29}

\section*{Монотонные и строго монотонные функции}

\subsection*{Монотонная функция}
Функция \( f(x) \) называется монотонной на промежутке \( I \), если она либо не убывает, либо не возрастает на этом промежутке.

\subsubsection*{Неубывающая функция}
Функция \( f(x) \) называется неубывающей на промежутке \( I \), если для любых \( x_1, x_2 \in I \) таких, что \( x_1 < x_2 \), выполняется неравенство:
\[
f(x_1) \leq f(x_2)
\]

\subsubsection*{Невозрастающая функция}
Функция \( f(x) \) называется невозрастающей на промежутке \( I \), если для любых \( x_1, x_2 \in I \) таких, что \( x_1 < x_2 \), выполняется неравенство:
\[
f(x_1) \geq f(x_2)
\]

\subsection*{Строго монотонная функция}
Функция \( f(x) \) называется строго монотонной на промежутке \( I \), если она либо строго возрастает, либо строго убывает на этом промежутке.

\subsubsection*{Строго возрастающая функция}
Функция \( f(x) \) называется строго возрастающей на промежутке \( I \), если для любых \( x_1, x_2 \in I \) таких, что \( x_1 < x_2 \), выполняется неравенство:
\[
f(x_1) < f(x_2)
\]

\subsubsection*{Строго убывающая функция}
Функция \( f(x) \) называется строго убывающей на промежутке \( I \), если для любых \( x_1, x_2 \in I \) таких, что \( x_1 < x_2 \), выполняется неравенство:
\[
f(x_1) > f(x_2)
\]

\section*{Теорема о непрерывности обратной функции}

\textbf{Теорема:} Пусть функция \( f(x) \) определена, строго возрастает (убывает) и непрерывна на отрезке \([a, b]\). Тогда обратная функция \( f^{-1}(y) \) определена, однозначна, строго возрастает (убывает) и непрерывна на отрезке с концами в точках \( f(a) \) и \( f(b) \).

\section{Билет 30}

\section*{Производная функции в точке. Односторонние производные}

\subsection{Определение 1. Производная функции в точке}

Пусть функция \( f(x) \) определена в некоторой окрестности точки \( x_0 \). Производной функции \( f(x) \) в точке \( x_0 \) называется предел отношения приращения функции к приращению аргумента, если этот предел существует:
\[
f'(x_0) = \lim\limits_{\Delta x \to 0} \frac{f(x_0 + \Delta x) - f(x_0)}{\Delta x}.
\]
Здесь \( \Delta x = x - x_0 \) — приращение аргумента.

\subsection{Определение 2. Односторонние производные}

Если предел существует только при одностороннем стремлении \( \Delta x \to 0^+ \) или \( \Delta x \to 0^- \), то говорят об односторонних производных. Они определяются следующим образом:
\[
f'_+(x_0) = \lim\limits_{\Delta x \to 0^+} \frac{f(x_0 + \Delta x) - f(x_0)}{\Delta x}, \quad
f'_-(x_0) = \lim\limits_{\Delta x \to 0^-} \frac{f(x_0 + \Delta x) - f(x_0)}{\Delta x}.
\]

\subsection{Теорема 1. Существование производной}

Если в точке \( x_0 \) существуют обе односторонние производные \( f'_+(x_0) \) и \( f'_-(x_0) \) и они равны, то существует производная \( f'(x_0) \), и она равна \( f'_+(x_0) = f'_-(x_0) \).

\subsection{Пример 1. Вычисление производной в точке}

Рассмотрим функцию \( f(x) = x^2 \). Найдем производную в точке \( x_0 = 1 \):
\[
f'(x_0) = \lim\limits_{\Delta x \to 0} \frac{f(x_0 + \Delta x) - f(x_0)}{\Delta x}.
\]
Подставим:
\[
f'(1) = \lim\limits_{\Delta x \to 0} \frac{(1 + \Delta x)^2 - 1^2}{\Delta x}.
\]
Раскроем скобки:
\[
f'(1) = \lim\limits_{\Delta x \to 0} \frac{1 + 2\Delta x + (\Delta x)^2 - 1}{\Delta x} = \lim\limits_{\Delta x \to 0} \frac{2\Delta x + (\Delta x)^2}{\Delta x}.
\]
Упростим:
\[
f'(1) = \lim\limits_{\Delta x \to 0} (2 + \Delta x) = 2.
\]
Таким образом, \( f'(1) = 2 \).

\subsection{Пример 2. Односторонние производные}

Рассмотрим функцию:
\[
f(x) =
\begin{cases}
x^2, & x \geq 0, \\
-x^2, & x < 0.
\end{cases}
\]
Найдем односторонние производные в точке \( x_0 = 0 \):
\[
f'_+(0) = \lim\limits_{\Delta x \to 0^+} \frac{f(0 + \Delta x) - f(0)}{\Delta x} = \lim\limits_{\Delta x \to 0^+} \frac{(\Delta x)^2 - 0}{\Delta x} = \lim\limits_{\Delta x \to 0^+} \Delta x = 0.
\]
\[
f'_-(0) = \lim\limits_{\Delta x \to 0^-} \frac{f(0 + \Delta x) - f(0)}{\Delta x} = \lim\limits_{\Delta x \to 0^-} \frac{- (\Delta x)^2 - 0}{\Delta x} = \lim\limits_{\Delta x \to 0^-} -\Delta x = 0.
\]
Поскольку \( f'_+(0) = f'_-(0) = 0 \), то \( f'(0) = 0 \).

\section{Билет 31}

\section*{Функция, дифференцируемая в точке}

\subsection{Определение. Дифференцируемость функции в точке}

Функция \( f(x) \), определённая в окрестности точки \( x_0 \), называется дифференцируемой в точке \( x_0 \), если её приращение \( \Delta y = f(x_0 + \Delta x) - f(x_0) \) может быть представлено в виде:
\[
\Delta y = A \Delta x + o(\Delta x), \quad \text{где } A \text{ — постоянная, а } o(\Delta x) \text{ — бесконечно малая величина}.
\]
При этом число \( A \) называется производной функции \( f(x) \) в точке \( x_0 \) и обозначается как \( f'(x_0) \).

\subsection{Теорема. Необходимое и достаточное условие дифференцируемости функции в точке}

Функция \( f(x) \), определённая в окрестности точки \( x_0 \), дифференцируема в точке \( x_0 \) тогда и только тогда, когда:
\begin{enumerate}
    \item функция \( f(x) \) непрерывна в точке \( x_0 \), то есть:
    \[
    \lim_{x \to x_0} f(x) = f(x_0);
    \]
    \item существует конечная производная:
    \[
    f'(x_0) = \lim_{\Delta x \to 0} \frac{\Delta y}{\Delta x},
    \]
    причём выполняется представление:
    \[
    \Delta y = f'(x_0) \Delta x + o(\Delta x).
    \]
\end{enumerate}

\subsection{Пример 1. Проверка дифференцируемости функции}

Рассмотрим функцию \( f(x) = x^2 \). Проверим, является ли она дифференцируемой в точке \( x_0 = 1 \).

Приращение функции:
\[
\Delta y = f(x_0 + \Delta x) - f(x_0) = (1 + \Delta x)^2 - 1^2 = 2\Delta x + (\Delta x)^2.
\]
Разделим \( \Delta y \) на \( \Delta x \):
\[
\frac{\Delta y}{\Delta x} = 2 + \Delta x.
\]
При \( \Delta x \to 0 \) получаем:
\[
f'(1) = \lim_{\Delta x \to 0} \frac{\Delta y}{\Delta x} = 2.
\]
Так как \( \Delta y = f'(1) \Delta x + o(\Delta x) \), функция \( f(x) = x^2 \) дифференцируема в точке \( x_0 = 1 \).

\subsection{Пример 2. Проверка необходимости непрерывности}

Рассмотрим функцию:
\[
f(x) = 
\begin{cases} 
x^2, & x \geq 0, \\ 
-x^2, & x < 0.
\end{cases}
\]
Для \( x_0 = 0 \) найдём производные слева и справа:
\[
f'_+(0) = \lim_{\Delta x \to 0^+} \frac{(0 + \Delta x)^2 - 0}{\Delta x} = \lim_{\Delta x \to 0^+} \Delta x = 0,
\]
\[
f'_-(0) = \lim_{\Delta x \to 0^-} \frac{-(0 + \Delta x)^2 - 0}{\Delta x} = \lim_{\Delta x \to 0^-} -\Delta x = 0.
\]
Производная существует, но:
\[
\lim_{x \to 0^-} f(x) = 0, \quad \lim_{x \to 0^+} f(x) = 0, \quad f(0) = 0.
\]
Функция \( f(x) \) непрерывна, а значит, и дифференцируема в \( x_0 = 0 \). Требование непрерывности выполняется.

\section{Билет 32}

\section*{Теорема о связи дифференцируемости функции в точке с непрерывностью в этой точке}

\subsection{Формулировка теоремы}
Если функция \( f(x) \) дифференцируема в точке \( x_0 \), то она непрерывна в этой точке.

\subsection{Доказательство}
Пусть функция \( f(x) \) дифференцируема в точке \( x_0 \). Тогда по определению дифференцируемости её приращение можно записать в виде:
\[
\Delta y = f'(x_0) \Delta x + o(\Delta x),
\]
где \( \Delta y = f(x_0 + \Delta x) - f(x_0) \).

Поделим обе части на \( \Delta x \) (при \( \Delta x \neq 0 \)):
\[
\frac{\Delta y}{\Delta x} = f'(x_0) + \frac{o(\Delta x)}{\Delta x}.
\]
Переходя к пределу при \( \Delta x \to 0 \), получаем:
\[
\lim_{\Delta x \to 0} \frac{\Delta y}{\Delta x} = f'(x_0).
\]
Так как \( o(\Delta x) \to 0 \) быстрее, чем \( \Delta x \), то при \( \Delta x \to 0 \):
\[
\Delta y \to 0 \quad \Rightarrow \quad f(x_0 + \Delta x) \to f(x_0).
\]
Это и означает, что функция \( f(x) \) непрерывна в точке \( x_0 \).

\subsection{Пример 1. Проверка теоремы на функции \( f(x) = x^2 \)}
Рассмотрим функцию \( f(x) = x^2 \). Найдём её производную в точке \( x_0 = 1 \):
\[
\Delta y = f(x_0 + \Delta x) - f(x_0) = (1 + \Delta x)^2 - 1^2 = 2\Delta x + (\Delta x)^2.
\]
Производная:
\[
\frac{\Delta y}{\Delta x} = 2 + \Delta x, \quad f'(1) = \lim_{\Delta x \to 0} \frac{\Delta y}{\Delta x} = 2.
\]
Так как \( \Delta y \to 0 \) при \( \Delta x \to 0 \), функция непрерывна в точке \( x_0 = 1 \).

\subsection{Пример 2. Проверка непрерывности при отсутствии дифференцируемости}
Рассмотрим функцию \( f(x) = |x| \). Её производная не существует в точке \( x_0 = 0 \), так как:
\[
\lim_{\Delta x \to 0^+} \frac{|\Delta x|}{\Delta x} = 1, \quad \lim_{\Delta x \to 0^-} \frac{|\Delta x|}{\Delta x} = -1.
\]
Однако функция \( f(x) = |x| \) непрерывна в точке \( x_0 = 0 \), так как:
\[
\lim_{x \to 0} f(x) = f(0) = 0.
\]
Это показывает, что непрерывность не является достаточным условием для дифференцируемости.


\section{Билет 33}

\section*{Понятие дифференциала функции}

\subsection{Определение}
Дифференциал функции \( f(x) \) в точке \( x_0 \) — это линейная часть приращения функции. Если функция \( f(x) \) дифференцируема в точке \( x_0 \), то дифференциал \( df \) определяется как:
\[
df = f'(x_0) dx,
\]
где \( dx \) — произвольное приращение аргумента.

\subsection{Пример}
Рассмотрим функцию \( f(x) = x^2 \). Её производная равна \( f'(x) = 2x \). Тогда дифференциал функции равен:
\[
df = 2x dx.
\]
Если \( x = 1 \) и \( dx = 0.1 \), то дифференциал равен:
\[
df = 2 \cdot 1 \cdot 0.1 = 0.2.
\]

\section{Билет 34}

\section*{Геометрический смысл производной и дифференциала, секущая и касательная к графику функции в данной точке}

\subsection{Геометрический смысл производной}
Производная функции \( f(x) \) в точке \( x_0 \) равна угловому коэффициенту касательной к графику функции в этой точке:
\[
\tan \alpha = f'(x_0),
\]
где \( \alpha \) — угол наклона касательной к оси \( x \).

\subsection{Геометрический смысл дифференциала}
Дифференциал \( df \) — это приращение ординаты касательной, соответствующее приращению \( dx \):
\[
df = f'(x_0) dx.
\]

\subsection{Уравнение касательной и нормали}
\subsubsection{Касательная}
Уравнение касательной к графику функции \( y = f(x) \) в точке \( x_0 \):
\[
y - f(x_0) = f'(x_0)(x - x_0).
\]
\subsubsection{Нормаль}
Уравнение нормали к графику функции \( y = f(x) \) в точке \( x_0 \):
\[
y - f(x_0) = -\frac{1}{f'(x_0)}(x - x_0).
\]

\subsection{Пример}
Рассмотрим функцию \( f(x) = x^2 \). В точке \( x_0 = 1 \):
\[
f'(x) = 2x, \quad f'(1) = 2.
\]
Уравнение касательной:
\[
y - 1 = 2(x - 1), \quad y = 2x - 1.
\]
Уравнение нормали:
\[
y - 1 = -\frac{1}{2}(x - 1), \quad y = -\frac{1}{2}x + \frac{3}{2}.
\]

\subsection{Уравнение касательной и нормали}

\section{Билет 35}

\section*{Теорема об арифметических действиях с дифференцируемыми функциями}

\subsection{Формулировка теоремы}
Пусть функции \( f(x) \) и \( g(x) \) дифференцируемы в точке \( x_0 \). Тогда:
\begin{enumerate}
    \item \( (f + g)'(x_0) = f'(x_0) + g'(x_0) \);
    \item \( (f - g)'(x_0) = f'(x_0) - g'(x_0) \);
    \item \( (cf)'(x_0) = cf'(x_0) \), где \( c \) — константа;
    \item \( (fg)'(x_0) = f'(x_0)g(x_0) + f(x_0)g'(x_0) \);
    \item \( \left(\frac{f}{g}\right)'(x_0) = \frac{f'(x_0)g(x_0) - f(x_0)g'(x_0)}{g^2(x_0)}, \quad g(x_0) \neq 0 \).
\end{enumerate}

\subsection{Доказательство}
Следует из определения производной и арифметических действий с пределами.

\subsection{Пример}
Рассмотрим функции \( f(x) = x^2 \) и \( g(x) = x + 1 \). Тогда:
\begin{enumerate}
    \item \( (f + g)'(x) = (x^2 + x + 1)' = 2x + 1 \);
    \item \( (fg)'(x) = (x^2(x + 1))' = 2x(x + 1) + x^2 = 3x^2 + 2x \).
\end{enumerate}

\section{Билет 36}

\section*{Теорема о производной сложной функции}

\subsection{Формулировка теоремы}
Если функция \( y = f(u) \) дифференцируема в точке \( u_0 \), а функция \( u = g(x) \) дифференцируема в точке \( x_0 \), то сложная функция \( y = f(g(x)) \) дифференцируема в точке \( x_0 \), и её производная равна:
\[
\frac{dy}{dx} = \frac{df}{du} \cdot \frac{du}{dx}.
\]

\subsection{Доказательство}
Следует из определения производной как предела и замены \( \Delta y = f(g(x + \Delta x)) - f(g(x)) \).

\subsection{Пример}
Рассмотрим функцию \( y = \sin(x^2) \). Тогда \( u = x^2 \) и \( f(u) = \sin u \). Найдём производную:
\[
\frac{dy}{dx} = \cos(x^2) \cdot 2x = 2x \cos(x^2).
\]

\subsection{Свойство инвариантности формы первого дифференциала}

\section{Билет 37}

\section*{Теорема о производной обратной функции}

\subsection{Формулировка теоремы}
Пусть функция \( y = f(x) \) монотонна и дифференцируема на промежутке, причём \( f'(x) \neq 0 \). Тогда обратная функция \( x = f^{-1}(y) \) дифференцируема, и её производная равна:
\[
\frac{dx}{dy} = \frac{1}{f'(x)}.
\]

\subsection{Пример}
Рассмотрим функцию \( y = x^3 \). Тогда обратная функция \( x = \sqrt[3]{y} \). Найдём её производную:
\[
\frac{dx}{dy} = \frac{1}{3x^2}.
\]

\section{Билет 38}

\section*{Таблица производных основных элементарных функций}

\[
\begin{array}{|c|c|}
\hline
\textbf{Функция} & \textbf{Производная} \\
\hline
c & 0 \\
x^n & n x^{n-1}, \, n \in \mathbb{R} \\
\sin x & \cos x \\
\cos x & -\sin x \\
\tan x & \sec^2 x, \, x \neq \frac{\pi}{2} + \pi k \, (k \in \mathbb{Z}) \\
\cot x & -\csc^2 x, \, x \neq \pi k \, (k \in \mathbb{Z}) \\
e^x & e^x \\
a^x & a^x \ln a, \, a > 0, \, a \neq 1 \\
\ln x & \frac{1}{x}, \, x > 0 \\
\log_a x & \frac{1}{x \ln a}, \, x > 0, \, a > 0, \, a \neq 1 \\
\hline
\end{array}
\]

\section{Билет 39}

\section*{Производные и дифференциалы высших порядков}

\subsection{Формула Лейбница для производной n-го порядка от произведения двух функций}

\subsection*{Формула Лейбница}
Пусть функции \( u(x) \) и \( v(x) \) имеют производные до порядка \( n \) включительно. Тогда производная \( n \)-го порядка от их произведения вычисляется по формуле:
\[
(u(x) \cdot v(x))^{(n)} = \sum_{k=0}^n \binom{n}{k} u^{(k)}(x) v^{(n-k)}(x),
\]
где \( \binom{n}{k} = \frac{n!}{k! (n-k)!} \) — биномиальный коэффициент.


\section{Билет 40}

\section*{Параметрический способ задания функции}
Функция может быть задана параметрически в виде:
\[
x = \varphi(t), \quad y = \psi(t), \quad t \in [a, b],
\]
где \( t \) — параметр. 

Производная функции \( y \) по \( x \) в таком случае вычисляется по формуле:
\[
\frac{dy}{dx} = \frac{\frac{dy}{dt}}{\frac{dx}{dt}}, \quad \text{где } \frac{dx}{dt} \neq 0.
\]

\section{Билет 41}

\section*{Теорема Ролля}
Пусть функция \( f(x) \):
\begin{enumerate}
    \item непрерывна на отрезке \( [a, b] \),
    \item дифференцируема на интервале \( (a, b) \),
    \item удовлетворяет условию \( f(a) = f(b) \).
\end{enumerate}
Тогда существует хотя бы одна точка \( c \in (a, b) \), такая что:
\[
f'(c) = 0.
\]

\section{Билет 42}

\section*{Теорема о среднем Лагранжа (формула Лагранжа)}
Пусть функция \( f(x) \):
\begin{enumerate}
    \item непрерывна на отрезке \( [a, b] \),
    \item дифференцируема на интервале \( (a, b) \).
\end{enumerate}
Тогда существует хотя бы одна точка \( c \in (a, b) \), такая что:
\[
f'(c) = \frac{f(b) - f(a)}{b - a}.
\]

\subsection*{Геометрическая интерпретация}
Точка \( c \) — это такая точка на интервале \( (a, b) \), в которой касательная к графику функции \( f(x) \) параллельна секущей, проходящей через точки \( (a, f(a)) \) и \( (b, f(b)) \).
\section*{Теорема о среднем Лагранжа (формула Лагранжа)}
Пусть функция \( f(x) \):
\begin{enumerate}
    \item непрерывна на отрезке \( [a, b] \),
    \item дифференцируема на интервале \( (a, b) \).
\end{enumerate}
Тогда существует хотя бы одна точка \( c \in (a, b) \), такая что:
\[
f'(c) = \frac{f(b) - f(a)}{b - a}.
\]

\subsection*{Геометрическая интерпретация}
Точка \( c \) — это такая точка на интервале \( (a, b) \), в которой касательная к графику функции \( f(x) \) параллельна секущей, проходящей через точки \( (a, f(a)) \) и \( (b, f(b)) \).


\subsection{Геометрическая интерпретация}

\section{Билет 43}

\section*{Теорема о среднем Коши}
Пусть функции \( f(x) \) и \( g(x) \):
\begin{enumerate}
    \item непрерывны на отрезке \( [a, b] \),
    \item дифференцируемы на интервале \( (a, b) \),
    \item \( g'(x) \neq 0 \) на \( (a, b) \).
\end{enumerate}
Тогда существует хотя бы одна точка \( c \in (a, b) \), такая что:
\[
\frac{f'(c)}{g'(c)} = \frac{f(b) - f(a)}{g(b) - g(a)}.
\]

\subsection*{Формула Коши}
Формула Коши позволяет найти соотношение между изменениями функций \( f(x) \) и \( g(x) \), используя производные в точке \( c \):
\[
f'(c) (g(b) - g(a)) = g'(c) (f(b) - f(a)).
\]

\section{Билет 44}

\section*{Правило Лопиталя}
Пусть функции \( f(x) \) и \( g(x) \):
\begin{enumerate}
    \item дифференцируемы в некоторой окрестности точки \( x_0 \) (за исключением, возможно, самой точки \( x_0 \)),
    \item \( g'(x) \neq 0 \) в этой окрестности.
\end{enumerate}
Если выполняется один из предельных случаев:
\[
\lim_{x \to x_0} \frac{f(x)}{g(x)} = \frac{0}{0} \quad \text{или} \quad \lim_{x \to x_0} \frac{f(x)}{g(x)} = \frac{\pm \infty}{\pm \infty},
\]
то:
\[
\lim_{x \to x_0} \frac{f(x)}{g(x)} = \lim_{x \to x_0} \frac{f'(x)}{g'(x)},
\]
если предел справа существует.

\section{Билет 45}

\section*{Формула Тейлора с остаточным членом в форме Пеано}
Если функция \( f(x) \) \( n \)-раз дифференцируема в точке \( x = a \), то она может быть представлена в окрестности \( a \) в виде:
\[
f(x) = P_n(x) + o((x-a)^n),
\]
где \( P_n(x) \) — многочлен Тейлора степени \( n \), определяемый как:
\[
P_n(x) = f(a) + f'(a)(x-a) + \frac{f''(a)}{2!}(x-a)^2 + \dots + \frac{f^{(n)}(a)}{n!}(x-a)^n.
\]

\subsection*{Остаточный член в форме Лагранжа}
Остаточный член может быть записан в форме:
\[
R_n(x) = \frac{f^{(n+1)}(\xi)}{(n+1)!}(x-a)^{n+1},
\]
где \( \xi \in (a, x) \).

\subsection*{Формула Маклорена}
Формула Маклорена — это частный случай формулы Тейлора, когда \( a = 0 \):
\[
f(x) = f(0) + f'(0)x + \frac{f''(0)}{2!}x^2 + \dots + \frac{f^{(n)}(0)}{n!}x^n + R_n(x).
\]

Примеры разложения элементарных функций:
\[
e^x = 1 + x + \frac{x^2}{2!} + \frac{x^3}{3!} + \dots,
\]
\[
\sin x = x - \frac{x^3}{3!} + \frac{x^5}{5!} - \dots,
\]
\[
\cos x = 1 - \frac{x^2}{2!} + \frac{x^4}{4!} - \dots.
\]


\subsection{ Формула Маклорена, разложение некоторых элементарных функций по формуле Маклорена}

\section{Билет 46}

\section*{Теорема о невозрастающей и неубывающей функции}
Пусть функция \( f(x) \) непрерывна на \( [a, b] \) и дифференцируема на \( (a, b) \). Тогда:
\begin{itemize}
    \item Если \( f'(x) \geq 0 \) на \( (a, b) \), то \( f(x) \) неубывает на \( [a, b] \).
    \item Если \( f'(x) \leq 0 \) на \( (a, b) \), то \( f(x) \) невозрастает на \( [a, b] \).
\end{itemize}

\subsection*{Теорема о достаточном условии возрастания или убывания функции в точке}
Если производная \( f'(x) > 0 \) в некоторой окрестности точки \( x_0 \), то функция \( f(x) \) возрастает в этой окрестности. Аналогично, если \( f'(x) < 0 \), то \( f(x) \) убывает.

\section{Билет 47}

\section*{Понятие локального экстремума функции}
Точка \( x_0 \) называется точкой локального экстремума функции \( f(x) \), если существует окрестность \( (x_0 - \delta, x_0 + \delta) \), такая что:
\begin{itemize}
    \item \( f(x_0) \) — локальный минимум, если \( f(x_0) \leq f(x) \) для всех \( x \) из этой окрестности.
    \item \( f(x_0) \) — локальный максимум, если \( f(x_0) \geq f(x) \) для всех \( x \) из этой окрестности.
\end{itemize}

\section{Билет 48}

\section*{Теорема о необходимом условии экстремума}
Если функция \( f(x) \) имеет локальный экстремум в точке \( x_0 \), и при этом \( f(x) \) дифференцируема в \( x_0 \), то её производная в этой точке равна нулю:
\[
f'(x_0) = 0.
\]

\subsection*{Замечание}
Условие \( f'(x_0) = 0 \) является необходимым, но не достаточным для экстремума. Такие точки \( x_0 \), в которых \( f'(x_0) = 0 \), называются критическими точками.


\section{Билет 49}

\section*{Теорема 1 (достаточное условие максимума или минимума)}
Пусть \( f(x) \) дважды дифференцируема в точке \( x_0 \):
\begin{itemize}
    \item Если \( f'(x_0) = 0 \) и \( f''(x_0) > 0 \), то \( x_0 \) — точка локального минимума.
    \item Если \( f'(x_0) = 0 \) и \( f''(x_0) < 0 \), то \( x_0 \) — точка локального максимума.
\end{itemize}

\section*{Теорема 2 (общее достаточное условие)}
Пусть \( f(x) \) \( n \)-раз дифференцируема в точке \( x_0 \), и пусть:
\begin{itemize}
    \item \( f'(x_0) = f''(x_0) = \dots = f^{(n-1)}(x_0) = 0 \),
    \item \( f^{(n)}(x_0) \neq 0 \), где \( n \geq 2 \).
\end{itemize}
Тогда:
\begin{itemize}
    \item Если \( n \) чётное и \( f^{(n)}(x_0) > 0 \), то \( x_0 \) — точка минимума.
    \item Если \( n \) чётное и \( f^{(n)}(x_0) < 0 \), то \( x_0 \) — точка максимума.
    \item Если \( n \) нечётное, то в точке \( x_0 \) экстремума нет.
\end{itemize}

\subsection*{Правила нахождения экстремумов}
\begin{enumerate}
    \item Найти производную \( f'(x) \) и определить критические точки, решив уравнение \( f'(x) = 0 \).
    \item Проверить знак второй производной \( f''(x) \) в критических точках для подтверждения характера экстремума.
\end{enumerate}

\section{Билет 50}

\section*{Понятие направления выпуклости графика функции}
Функция \( f(x) \) называется выпуклой вверх на интервале \( (a, b) \), если:
\[
f''(x) > 0 \quad \forall x \in (a, b).
\]
Функция \( f(x) \) называется выпуклой вниз на интервале \( (a, b) \), если:
\[
f''(x) < 0 \quad \forall x \in (a, b).
\]

\subsection*{Определение точки перегиба}
Точка \( x_0 \) называется точкой перегиба графика функции \( f(x) \), если в этой точке функция \( f(x) \) меняет направление выпуклости:
\[
f''(x) = 0, \quad \text{и знак } f''(x) \text{ меняется в точке } x_0.
\]

\subsection*{Теорема о достаточных условиях перегиба графика функции}
Если \( f(x) \) трижды дифференцируема в точке \( x_0 \), и:
\begin{itemize}
    \item \( f''(x_0) = 0 \),
    \item \( f'''(x_0) \neq 0 \),
\end{itemize}
то \( x_0 \) — точка перегиба графика функции \( f(x) \).


\section{Билет 51}

\section*{Определение вертикальной асимптоты}
Прямая \( x = a \) называется вертикальной асимптотой графика функции \( f(x) \), если хотя бы один из пределов:
\[
\lim_{x \to a^-} f(x) = \pm \infty \quad \text{или} \quad \lim_{x \to a^+} f(x) = \pm \infty
\]
существует.

\subsection*{Наклонные и горизонтальные асимптоты}
\begin{itemize}
    \item Прямая \( y = b \) называется горизонтальной асимптотой, если:
    \[
    \lim_{x \to \pm \infty} f(x) = b.
    \]
    \item Прямая \( y = kx + b \) называется наклонной асимптотой, если:
    \[
    \lim_{x \to \pm \infty} \left( f(x) - (kx + b) \right) = 0.
    \]
\end{itemize}


\section{Билет 52}

\section*{Понятие первообразной}

\subsection*{Определение 1}
Функция \( F(x) \) называется первообразной функции \( f(x) \) на промежутке \( X \), если для любого \( x \in X \) выполняется равенство:
\[
F'(x) = f(x).
\]

\subsection*{Теорема 1}
Если \( F(x) \) является первообразной функции \( f(x) \) на промежутке \( X \), то множество всех первообразных \( f(x) \) на \( X \) имеет вид:
\[
F(x) + C,
\]
где \( C \) — произвольная постоянная.

\subsection*{Определение 2}
Совокупность всех первообразных функции \( f(x) \) на промежутке \( X \) называется неопределённым интегралом функции \( f(x) \) и обозначается:
\[
\int f(x) \, dx.
\]

\subsection*{Определение 3}
Операция нахождения первообразной функции \( f(x) \), или нахождение неопределённого интеграла, называется интегрированием.

\subsection*{Запись неопределённого интеграла}
Неопределённый интеграл записывается в виде:
\[
\int f(x) \, dx = F(x) + C,
\]
где \( F(x) \) — первообразная, а \( C \) — произвольная постоянная интегрирования.


\section{Билет 53}

\section*{Основные свойства неопределённого интеграла}
Пусть \( \int f(x) \, dx = F(x) + C \) и \( \int g(x) \, dx = G(x) + C \). Тогда:
\begin{enumerate}
    \item \textbf{Линейность:}
    \[
    \int \left( af(x) + bg(x) \right) \, dx = a \int f(x) \, dx + b \int g(x) \, dx,
    \]
    где \( a, b \) — постоянные.
    \item \textbf{Интеграл от нуля:}
    \[
    \int 0 \, dx = C.
    \]
    \item \textbf{Интеграл от производной:}
    \[
    \int f'(x) \, dx = f(x) + C.
    \]
\end{enumerate}

\section*{Метод интегрирования подстановкой}
Если \( x = \phi(t) \) — дифференцируемая функция, то:
\[
\int f(x) \, dx = \int f(\phi(t)) \phi'(t) \, dt.
\]
\subsection*{Пример}
Вычислить \( \int x \cos(x^2) \, dx \).
\[
\text{Пусть } u = x^2 \implies du = 2x \, dx.
\]
Тогда:
\[
\int x \cos(x^2) \, dx = \frac{1}{2} \int \cos(u) \, du = \frac{1}{2} \sin(u) + C = \frac{1}{2} \sin(x^2) + C.
\]

\section*{Интегрирование по частям}
Если \( u(x) \) и \( v(x) \) дифференцируемы, то:
\[
\int u(x) v'(x) \, dx = u(x) v(x) - \int v(x) u'(x) \, dx.
\]
\subsection*{Пример}
Вычислить \( \int x e^x \, dx \).
\[
\text{Пусть } u = x, \, dv = e^x \, dx \implies du = dx, \, v = e^x.
\]
Тогда:
\[
\int x e^x \, dx = x e^x - \int e^x \, dx = x e^x - e^x + C = e^x (x - 1) + C.
\]

\section*{Понятие многочлена}
Многочлен (полином) от одной переменной \( x \) степени \( n \) имеет вид:
\[
P(x) = a_n x^n + a_{n-1} x^{n-1} + \dots + a_1 x + a_0,
\]
где \( a_n, a_{n-1}, \dots, a_0 \) — коэффициенты многочлена, \( a_n \neq 0 \).

\subsection*{Сумма и произведение многочленов}
\begin{itemize}
    \item Сумма многочленов \( P(x) \) и \( Q(x) \) есть многочлен:
    \[
    (P + Q)(x) = P(x) + Q(x).
    \]
    \item Произведение многочленов \( P(x) \) и \( Q(x) \) есть многочлен:
    \[
    (P \cdot Q)(x) = P(x) \cdot Q(x).
    \]
\end{itemize}

\subsection*{Деление многочленов}
Каждый многочлен \( P(x) \) можно представить в виде:
\[
P(x) = Q(x) \cdot D(x) + R(x),
\]
где \( Q(x) \) — частное, \( R(x) \) — остаток, причём \( \deg R(x) < \deg D(x) \).

\subsection*{Теорема Безу}
Если \( P(a) = 0 \), то \( x - a \) является делителем многочлена \( P(x) \). Более того:
\[
P(x) = (x - a) Q(x),
\]
где \( Q(x) \) — частное от деления \( P(x) \) на \( x - a \).


\section{Билет 57}

\section*{Основная теорема высшей алгебры}
\textbf{Теорема.} Любой многочлен \( P(x) \) степени \( n \geq 1 \) с комплексными коэффициентами имеет хотя бы один корень в поле комплексных чисел \( \mathbb{C} \).

\subsection*{Следствие}
Любой многочлен степени \( n \) с комплексными коэффициентами представим в виде:
\[
P(x) = a_n (x - x_1)(x - x_2) \cdots (x - x_n),
\]
где \( x_1, x_2, \dots, x_n \) — корни многочлена \( P(x) \) (возможно, кратные), а \( a_n \) — старший коэффициент.

\subsection*{Свойства корней многочленов с вещественными коэффициентами}
Если многочлен \( P(x) \) имеет вещественные коэффициенты, то:
\begin{enumerate}
    \item Корни комплексно-сопряжённые. Если \( z = a + bi \) (\( b \neq 0 \)) — корень, то \( \overline{z} = a - bi \) также корень.
    \item Если степень \( n \) нечётная, то \( P(x) \) имеет хотя бы один вещественный корень.
\end{enumerate}

\subsection*{Разложение на неприводимые множители}
Любой многочлен \( P(x) \) с вещественными коэффициентами представим в виде произведения:
\[
P(x) = a_n (x - x_1)(x - x_2) \cdots (x - x_k) Q(x),
\]
где \( x_1, \dots, x_k \) — вещественные корни, а \( Q(x) \) — произведение неприводимых квадратных множителей:
\[
Q(x) = (x^2 + px + q),
\]
где \( p^2 - 4q < 0 \).

\section*{Понятие рациональной дроби}
\textbf{Определение.} Рациональной дробью называется выражение вида:
\[
R(x) = \frac{P(x)}{Q(x)},
\]
где \( P(x) \) и \( Q(x) \) — многочлены, причём \( Q(x) \neq 0 \).

\subsection*{Правильная и неправильная рациональные дроби}
\begin{itemize}
    \item \textbf{Правильная рациональная дробь:} \( \deg P(x) < \deg Q(x) \).
    \item \textbf{Неправильная рациональная дробь:} \( \deg P(x) \geq \deg Q(x) \). Любую неправильную дробь можно представить в виде суммы:
    \[
    R(x) = Q(x) + \frac{P_1(x)}{Q(x)},
    \]
    где \( Q(x) \) — частное, а \( \frac{P_1(x)}{Q(x)} \) — правильная дробь.
\end{itemize}

\subsection*{Теорема о разложении правильной рациональной дроби в сумму простейших дробей}
\textbf{Теорема.} Любая правильная рациональная дробь \( R(x) = \frac{P(x)}{Q(x)} \), где \( Q(x) \) разложим на неприводимые множители, представима в виде суммы простейших дробей:
\[
R(x) = \sum_{i} \frac{A_i}{(x - a_i)^{m_i}} + \sum_{j} \frac{B_j x + C_j}{(x^2 + p_j x + q_j)^{n_j}},
\]
где:
\begin{itemize}
    \item \( x - a_i \) — линейные множители \( Q(x) \);
    \item \( x^2 + p_j x + q_j \) — неприводимые квадратичные множители \( Q(x) \);
    \item \( A_i, B_j, C_j \) — постоянные, определяемые из условий разложения.
\end{itemize}

\section*{Интегрирование дробно-рациональных выражений}
\textbf{Определение.} Интеграл от дробно-рационального выражения \( R(x) = \frac{P(x)}{Q(x)} \), где \( P(x) \) и \( Q(x) \) — многочлены, вычисляется путём разложения на простейшие дроби.

\subsection*{Методы интегрирования}
\begin{enumerate}
    \item \textbf{Разложение на простейшие дроби:} Представить \( R(x) \) в виде суммы дробей вида:
    \[
    \frac{A}{x - a}, \quad \frac{B x + C}{x^2 + px + q}, \quad \frac{A}{(x - a)^n}.
    \]
    Затем интегрировать каждую дробь отдельно.

    \item \textbf{Замена переменных:} При интегрировании выражений, содержащих квадратичные множители, полезно использовать подстановку:
    \[
    x = a \sin(t), \quad x = a \cosh(t), \quad x = \tan(t), \text{ и др.}
    \]

    \item \textbf{Интегрирование по частям:} Применимо, если дробь содержит логарифмические или обратные тригонометрические функции.
\end{enumerate}

\subsection*{Пример}
Вычислить \( \int \frac{1}{x^2 + 4} \, dx \).
\[
\text{Пусть } x^2 + 4 = 4(1 + \frac{x^2}{4}), \, x = 2 \tan(t) \implies dx = 2 \sec^2(t) \, dt.
\]
Тогда:
\[
\int \frac{1}{x^2 + 4} \, dx = \int \frac{1}{4 \sec^2(t)} \cdot 2 \sec^2(t) \, dt = \frac{1}{2} \int dt = \frac{1}{2} t + C.
\]
Возвращая замену, получаем:
\[
t = \arctan\left(\frac{x}{2}\right) \implies \int \frac{1}{x^2 + 4} \, dx = \frac{1}{2} \arctan\left(\frac{x}{2}\right) + C.
\]


\end{document}